% Copyright 2009-2011 Dominik Wagenfuehr <dominik.wagenfuehr@deesaster.org>
% Dieses Dokument unterliegt der Creative-Commons-Lizenz
% "Namensnennung-Weitergabe unter gleichen Bedingungen 3.0 Deutschland"
% [http://creativecommons.org/licenses/by-sa/3.0/de/].

% Pakete
\usepackage[utf8]{inputenc}   % UTF8-Kodierung für Umlaute
\usepackage[T1]{fontenc}      % use TeX encoding then Type 1.
\usepackage{lmodern}          % Bessere Schriftart (ersetzt CM-Schriften)
\usepackage{ngerman}          % deutsche Silbentrennung
\usepackage{graphicx}         % Einbindung von Grafiken
\usepackage{fancyhdr}         % eigenes Layout einbinden
\usepackage{pdfpages}         % PDF-Seiten einbinden
\usepackage{xifthen}          % Wenn-dann-Abfragen
\usepackage{charter}          % Charter-Schrift
\usepackage{titlesec}        % Anpassung der Überschriften
\usepackage{longtable}        % Tabellen über Seitenumbruch hinweg
\usepackage{setspace}         % Zeilenabstand festlegen
\usepackage{hyperref}         % Hyperlinks und interne PDF-Verweise
\usepackage{color}
\usepackage{marvosym}

\definecolor{gruen}{rgb}{0.30,0.65,0.15}

% Layout
\hbadness=10000                % unterdrückt unwichtige Fehlermeldungen
\clubpenalty = 10000           % Keine "Schusterjungen"
\widowpenalty = 10000          % Keine "Hurenkinder"
\displaywidowpenalty = 10000

% Linie im Kopf ausblenden und im Fuß eine feine Linie
\renewcommand*{\headrulewidth}{0pt}
\renewcommand*{\footrulewidth}{0.4pt}

% Überschriftenlayout verändern
\titlespacing{\section}{0mm}{2em}{2em}
\titlespacing{\subsection}{0em}{0em}{0em}
\titleformat{\section}{\normalfont\LARGE\scshape}{}{0mm}{\hspace*{\fill}}
\titleformat{\subsection}{\normalfont\Large\bfseries\scshape}{}{0mm}{}[\vskip-1.5em\hrulefill]

% Eigene Fusszeile festlegen, Kopfzeile leeren
\pagestyle{fancy}
%\fancyhead{}
\cfoot{\VollerName, \AbsenderStrasse, \AbsenderPLZOrt \\ \EMail, \Telefonnr, \Mobile}

% Absender
\newcommand{\Absender}[1][\normalsize]{%
    \rightline{\textbf{#1\VollerName}}
    \rightline{\AbsenderStrasse}
    \rightline{\AbsenderPLZOrt}
%    \rightline{\Telefonnr}
%    \rightline{\Mobile}
%    \rightline{\EMail}
}

% Adressat
\newcommand{\Anschrift}{%
    \textbf{\Firma}\\
    \AnredeKopf{} \AdressatVorname{} \AdressatNachname\\
    \AnschriftStrasse\\
    \AnschriftPLZOrt\\
}

% Makro für Ort und Unterschrift
\newcommand*{\UnterschriftOrt}{%
    \Unterschrift\\
    \OrtDatum\\
}

% Unterschrift als Bilddatei einfügen
\newcommand*{\Unterschrift}[1][0.1cm]{%
    \hspace*{#1}%&\includegraphics{\UnterschriftenDatei}%
}

% Tabelle für Lebenslauf
\newlength{\AbstandAbschnitt}
\newboolean{UnterabschnittBegonnen}
\newenvironment{Abschnitt}[1][0em]
{%
    \setlength{\AbstandAbschnitt}{#1}%
    \begin{longtable}{p{0.3\linewidth}p{0.64\linewidth}}%
    \setboolean{UnterabschnittBegonnen}{false}%
}
{%
    \end{longtable}%
    \vspace{\AbstandAbschnitt}%
}

% Unterabschnitt in Lebenslauftabelle mit sonstigen Themen
% Achtung, keine Umgebung, sondern ein Kommando!
\newlength{\AbstandUnterabschnitt}
\newcommand{\Unterabschnitt}[3][0em]
{%
    \end{longtable}%
    \ifthenelse{\boolean{UnterabschnittBegonnen}}{%
        \setlength{\AbstandUnterabschnitt}{-5em}%
    }{%
        \setlength{\AbstandUnterabschnitt}{-2em}%
    }%
    \addtolength{\AbstandUnterabschnitt}{#1}%
    \vspace{\AbstandUnterabschnitt}%
    \textit{#2} #3%
    \setboolean{UnterabschnittBegonnen}{true}%
    \begin{longtable}{p{0.3\linewidth}p{0.64\linewidth}}%
    \\[1em]
}

% Anrede im Brief
\newcommand*{\Anrede}{Sehr geehrte\AnredeText{} \AdressatNachname,
}

% Die Grußformel am Anschreibenende
\newcommand{\Grussworte}{%
\vspace{0.3cm}

mit freundlichen Grüßen,

\Unterschrift\\[1em]
}

% Betreff
\newcommand*{\Betreff}{\textcolor{gruen}{\textbf{\Bewerberstelle}}}

% Etwa Abstand vor dem Ort, um die Seite besser aufteilen zu können.
\newlength{\AbstandVorOrt}
\setlength{\AbstandVorOrt}{1em}

% Abstand zwischen den Adressen im Anschreiben.
\newlength{\AbstandZwischenAdressen}
\setlength{\AbstandZwischenAdressen}{0em}

\newcommand{\AnschreibenKopf}
{%
    % Anschreiben ohne Fuss- oder Kopfzeile
    \thispagestyle{empty}

    % Die Anschreibeseite wird bei Bedarf etwas vergrößert, da es hier keine
    % Fußzeile gibt.
    % \enlargethispage{3cm}
    
    \begin{spacing}{1.05}
        % Meine Anschrift
        \Absender

        % ggf. will man etwas Abstand zwischen den Adressen.
        \vspace*{\AbstandZwischenAdressen}

        % Anschrift der Firma
        \Anschrift

        % ggf. braucht man hier etwas Abstand, je nachdem, wie
        % viel Text man im Anschreiben hat.
        \vspace*{\AbstandVorOrt}

        % Ort und Datum
        \rightline{\OrtDatum}
    \end{spacing}

}

\newcommand{\Anschreiben}
{%
    \AnschreibenKopf
    \begin{spacing}{1.15}
        \Betreff
        
        %\begin{flushleft}
            \Anrede

            \AnschreibenText

            \Grussworte

            \AnschreibenAnlage
        %\end{flushleft}
    \end{spacing} 
}

\newcommand{\MeineSeite}
{%
    \section{Zu meiner Person}

    \begin{spacing}{1.3}
        \Absender[\large]
        \vskip2em

        \rightline{geboren am \Geburtstag}
        \rightline{in \Geburtsort}
        \vskip1em

        \rightline{\Details}

        \rule{10cm}{0.6pt} \\
        {\large \textbf{Ausbildungsgrad}}\\
        \Ausbildungsgrad\\

        %\includegraphics[width=5cm]{\BewerberFoto}\\

    \end{spacing}
}

\newcommand{\MeineMotivation}
{%
    \section{Über mich und meine Motivation}
 
    \begin{spacing}{1.3}
        \begin{flushleft}

        \Motivationstext

        \end{flushleft}
    \end{spacing}
    
    \UnterschriftOrt
}
